\documentclass[journal,12pt,twocolumn]{IEEEtran}
%
\usepackage{setspace}
\usepackage{gensymb}
%\doublespacing
\singlespacing

%\usepackage{graphicx}
%\usepackage{amssymb}
%\usepackage{relsize}
\usepackage[cmex10]{amsmath}
%\usepackage{amsthm}
%\interdisplaylinepenalty=2500
%\savesymbol{iint}
%\usepackage{txfonts}
%\restoresymbol{TXF}{iint}
%\usepackage{wasysym}
\usepackage{amsthm}
%\usepackage{iithtlc}
\usepackage{mathrsfs}
\usepackage{txfonts}
\usepackage{stfloats}
\usepackage{bm}
\usepackage{cite}
\usepackage{cases}
\usepackage{subfig}
%\usepackage{xtab}
\usepackage{longtable}
\usepackage{multirow}
%\usepackage{algorithm}
%\usepackage{algpseudocode}
\usepackage{enumitem}
\usepackage{mathtools}
\usepackage{tikz}
\usepackage{circuitikz}
\usepackage{verbatim}
%\usepackage{tfrupee}
\usepackage[breaklinks=true]{hyperref}
%\usepackage{stmaryrd}
\usepackage{tkz-euclide} % loads  TikZ and tkz-base
%\usetkzobj{all}
\usepackage{listings}
    \usepackage{color}                                            %%
    \usepackage{array}                                            %%
    \usepackage{longtable}                                        %%
    \usepackage{calc}                                             %%
    \usepackage{multirow}                                         %%
    \usepackage{hhline}                                           %%
    \usepackage{ifthen}                                           %%
  %optionally (for landscape tables embedded in another document): %%
    \usepackage{lscape}     
\usepackage{multicol}
\usepackage{chngcntr}
%\usepackage{enumerate}
\usepackage{circuitikz}
\usepackage{karnaugh-map}
\usetikzlibrary{calc}
%\usepackage{wasysym}
%\newcounter{MYtempeqncnt}
\DeclareMathOperator*{\Res}{Res}
%\renewcommand{\baselinestretch}{2}
\renewcommand\thesection{\arabic{section}}
\renewcommand\thesubsection{\thesection.\arabic{subsection}}
\renewcommand\thesubsubsection{\thesubsection.\arabic{subsubsection}}

\renewcommand\thesectiondis{\arabic{section}}
\renewcommand\thesubsectiondis{\thesectiondis.\arabic{subsection}}
\renewcommand\thesubsubsectiondis{\thesubsectiondis.\arabic{subsubsection}}

% correct bad hyphenation here
\hyphenation{op-tical net-works semi-conduc-tor}
\def\inputGnumericTable{}                                 %%

\lstset{
%language=C,
frame=single, 
breaklines=true,
columns=fullflexible
}
%\lstset{
%language=tex,
%frame=single, 
%breaklines=true
%}

\begin{document}
%


\newtheorem{theorem}{Theorem}[section]
\newtheorem{problem}{Problem}
\newtheorem{proposition}{Proposition}[section]
\newtheorem{lemma}{Lemma}[section]
\newtheorem{corollary}[theorem]{Corollary}
\newtheorem{example}{Example}[section]
\newtheorem{definition}[problem]{Definition}
%\newtheorem{thm}{Theorem}[section] 
%\newtheorem{defn}[thm]{Definition}
%\newtheorem{algorithm}{Algorithm}[section]
%\newtheorem{cor}{Corollary}
\newcommand{\BEQA}{\begin{eqnarray}}
\newcommand{\EEQA}{\end{eqnarray}}
\newcommand{\define}{\stackrel{\triangle}{=}}
\bibliographystyle{IEEEtran}
%\bibliographystyle{ieeetr}
\providecommand{\mbf}{\mathbf}
\providecommand{\pr}[1]{\ensuremath{\Pr\left(#1\right)}}
\providecommand{\qfunc}[1]{\ensuremath{Q\left(#1\right)}}
\providecommand{\sbrak}[1]{\ensuremath{{}\left[#1\right]}}
\providecommand{\lsbrak}[1]{\ensuremath{{}\left[#1\right.}}
\providecommand{\rsbrak}[1]{\ensuremath{{}\left.#1\right]}}
\providecommand{\brak}[1]{\ensuremath{\left(#1\right)}}
\providecommand{\lbrak}[1]{\ensuremath{\left(#1\right.}}
\providecommand{\rbrak}[1]{\ensuremath{\left.#1\right)}}
\providecommand{\cbrak}[1]{\ensuremath{\left\{#1\right\}}}
\providecommand{\lcbrak}[1]{\ensuremath{\left\{#1\right.}}
\providecommand{\rcbrak}[1]{\ensuremath{\left.#1\right\}}}
\theoremstyle{remark}
\newtheorem{rem}{Remark}
\newcommand{\sgn}{\mathop{\mathrm{sgn}}}
%\providecommand{\abs}[1]{\left\vert#1\right\vert}
\providecommand{\res}[1]{\Res\displaylimits_{#1}} 
%\providecommand{\norm}[1]{\left\lVert#1\right\rVert}
%\providecommand{\norm}[1]{\lVert#1\rVert}
\providecommand{\mtx}[1]{\mathbf{#1}}
%\providecommand{\mean}[1]{E\left[ #1 \right]}
\providecommand{\fourier}{\overset{\mathcal{F}}{ \rightleftharpoons}}
%\providecommand{\hilbert}{\overset{\mathcal{H}}{ \rightleftharpoons}}
\providecommand{\system}{\overset{\mathcal{H}}{ \longleftrightarrow}}
	%\newcommand{\solution}[2]{\textbf{Solution:}{#1}}
\newcommand{\solution}{\noindent \textbf{Solution: }}
\newcommand{\cosec}{\,\text{cosec}\,}
\providecommand{\dec}[2]{\ensuremath{\overset{#1}{\underset{#2}{\gtrless}}}}
\newcommand{\myvec}[1]{\ensuremath{\begin{pmatrix}#1\end{pmatrix}}}
\newcommand{\mydet}[1]{\ensuremath{\begin{vmatrix}#1\end{vmatrix}}}
\makeatletter
\@addtoreset{figure}{problem}
\makeatother
\let\StandardTheFigure\thefigure
\let\vec\mathbf
%\renewcommand{\thefigure}{\theproblem.\arabic{figure}}
\renewcommand{\thefigure}{\theproblem}
%\setlist[enumerate,1]{before=\renewcommand\theequation{\theenumi.\arabic{equation}}
%\counterwithin{equation}{enumi}
%\renewcommand{\theequation}{\arabic{subsection}.\arabic{equation}}
\def\putbox#1#2#3{\makebox[0in][l]{\makebox[#1][l]{}\raisebox{\baselineskip}[0in][0in]{\raisebox{#2}[0in][0in]{#3}}}}
     \def\rightbox#1{\makebox[0in][r]{#1}}
     \def\centbox#1{\makebox[0in]{#1}}
     \def\topbox#1{\raisebox{-\baselineskip}[0in][0in]{#1}}
     \def\midbox#1{\raisebox{-0.5\baselineskip}[0in][0in]{#1}}
\vspace{3cm}
\title{
%	\logo{
Assignment 1
%	}
}
\author{ Jagadeesh Kunuru}	
% make the title area
\maketitle
\newpage
%\tableofcontents
\bigskip
\renewcommand{\thefigure}{\theenumi}
\renewcommand{\thetable}{\theenumi}
%\renewcommand{\theequation}{\theenumi}
%Download the codes from
%\begin{lstlisting}
%$https://github.com/satyam463/FPGALAB/tree/main/Assignment1
%\end{lstlisting}
\section{Problem Statement}
\subsection{Icse/Cs/2019Q5c}

Using a truth table, state whether the following proposition is a Tautology, Contradiction or Contingency:

\begin{center}
       $\neg  ( P \Rightarrow Q ) \iff (\neg P \cup  Q)$
\end{center}
\section{Solution}
By drawing a truth table, we get to know whether the following proposition is Tautology, Contradiction or Contingency:

\vspace{15pt}

\begin{center}
\begin{tabular}{|c|c|c|c|c|c|c|c|} 
 \hline
 $P & Q$ & $\neg P$ & $P \Rightarrow Q$ & $\neg (P \Rightarrow Q )$ &  $(\neg P \cup Q)$ & $\neg (P \Rightarrow Q )$ \iff  $(\neg P \cup Q)$ \\ 
 T & T & F & T & F & T & F \\ 
 T & F & F & F & T & F & F \\ 
 F & T & T & T & F & T & F \\
 F & F & T & T & F & T & F \\
 \hline
\end{tabular}
\end{center}

\vspace{15pt}
\begin{center}
 Therefore, the given proposition is a Contradiction.

\vspace{15pt}

{\large Writing the Proposition in Boolean expression: }

\vspace{15pt}

K-Map for $\neg (P \Rightarrow Q )$

\vspace{15pt}

\begin{karnaugh-map}[2][2][1][$Q$][$P$]
        \minterms{2}
        \maxterms{0,1,3}
        %\implicant{1}{1}
        \implicant{2}{2}
    \end{karnaugh-map}
    
\vspace{10pt}

from K-Map
       $\neg (P \Rightarrow Q ) = P{Q}^{'}$
       
\vspace{50pt}
\begin{center}
K-Map for $(\neg P \cup Q)$

\vspace{15pt}

    
\begin{karnaugh-map}[2][2][1][$Q$][$P$]
        \minterms{0,1,3}
        \maxterms{2}
        \implicant{0}{1}
        \implicant{1}{3}
    \end{karnaugh-map}

\vspace{5pt}

from K-Map
       $(\neg P \cup Q) = {P}^{'}+Q$
\end{center}

\vspace{100pt}

Now, the proposition $\neg (P \Rightarrow Q )$ \iff  $(\neg P \cup Q)$  can written be as $P{Q}^{'}$\cap ${P}^{'}+Q$

Truth table for the Boolean expression:
\vspace{15pt}
\begin{center}
\begin{tabular}{|c|c|c|} 
 \hline
 $P{Q}^{'}$ & ${P}^{'}+Q$ & $P{Q}^{'}$\cap ${P}^{'}+Q$\\ 
 0 & 1 & 0\\ 
 1 & 0 & 0\\ 
 0 & 1 & 0\\
 0 & 1 & 0 \\
 \hline
\end{tabular}
\end{center}
\vspace{15pt}
$P{Q}^{'}{.}\brak{{P}^{'}+Q} = P{P}^{'}{Q}^{'}+{P}^{'}{Q}^{'}Q$ 
can be drawn as:\\
\vspace{15pt}
\begin{tikzpicture}
% Circuit style
\ctikzset{
    logic ports=ieee
}
% Logic ports
\node[nand port](NANDa) at (-1.5,0){};
\node[nand port](NANDb) at (-1.5,-1.5){};
\node[nand port](NANDc) at (1,-1){};
 \draw (NANDa.out) -| (NANDc.in 1);
\draw (NANDb.out) -| (NANDc.in 2);
\draw (NANDa.in 1) -- ++(-1.5,0)node[left](P){P};
\draw (NANDa.in 2) -- ++(-1.5,0)node[left]($P^{'}$){$P^{'}$};
\draw (NANDa.in 3) -- ++(-1.5,0)node[left]($Q^{'}$){$Q^{'}$};

\draw (NANDb.in 1) -- ++(-1.5,-1.5)node[left]($P^{'}$){$P^{'}$};
\draw (NANDb.in 2) -- ++(-1.5,-1.5)node[left]($Q^{'}$){$Q^{'}$};
\draw (NANDb.in 3) -- ++(-1.5,-1.5)node[left](Q){Q};
\draw (NANDc.out) -- ++(1,-1) node[near end,above]{F};
\end{tikzpicture}
\end{center}
\end{document}
